
\begin{DoxyEnumerate}
\item How to use
\end{DoxyEnumerate}
\begin{DoxyEnumerate}
\item To do list
\begin{DoxyEnumerate}
\item K\+L\+Tトラッカー試す
\end{DoxyEnumerate}
\begin{DoxyEnumerate}
\item 探索範囲制限したマッチング試す
\end{DoxyEnumerate}
\begin{DoxyEnumerate}
\item Fandmental Matrix求めて、エピポーラ線可視化してみる。
\end{DoxyEnumerate}
\end{DoxyEnumerate}
\begin{DoxyEnumerate}
\item Idea
\begin{DoxyEnumerate}
\item クリックだとかなりめんどくさいし意外と精度悪い ~\newline
 --$>$ クリックで周辺の領域指定、その中で\+A\+K\+A\+Z\+E、 指定領域でトラッキングして行く
\end{DoxyEnumerate}
\begin{DoxyEnumerate}
\item ランナーの足設置箇所判定 ~\newline
 ランナーとトラックを紐付け 足のエッジがトラックに現れた時に設置と判定
\end{DoxyEnumerate}
\begin{DoxyEnumerate}
\item S画像からトラック領域特定できるかも
\end{DoxyEnumerate}
\begin{DoxyEnumerate}
\item トラック外側領域を決定したら、外ラインの各点から 内ラインに下ろした垂線上で、白色領域をチェック。 その間隔から直線推定
\end{DoxyEnumerate}
\begin{DoxyEnumerate}
\item 各直線をトラッキングできたら、直線と各関節との距離から ランナーの歩幅を周期とするようなグラフが得られる
\end{DoxyEnumerate}

やること image numを自動で 読み込む画像の範囲を手動でできるように 
\end{DoxyEnumerate}